\chapter{Introduction}

\section{Context}

The development of enginnering application emplies the design
from strcuture to material.
Some how we must solve
we look for important materials.

what do we want to do?

why modeling is important?
How to model?

\section{Motivation}

In the context of produc-material-life we focus on the multi-physics and semii-crystalline.

\subsection{Multi-physics simulations}

Thermomechanics is widely acknowledged as a crucial physical phenomenon in engineering applications and a highly sought effect in computational models.
Thermomechanical interaction is key to adequately describe a large variety of technological processes, including sintering and material removal procedures.
To include it in a numerical simulation toolbox, the problem has to be correctly formulated in a thermodynamically consistent way.
Having achieved an appropriate formulation of the problem, the Finite Element Method can be used to solve it.

After the spatial discretization obtained by the FEM has been provided, the complete thermomechanical problem can still be solved by employing two different strategies: a monolithic approach or a partitioned approach.
In the former, the discretized equations representing the momentum balance and energy conservation are solved simultaneously.
In the latter, the problems are solved independently, solving the mechanical problem at a fixed temperature and then the thermal problem at a fixed configuration, taking the so-called isothermal split as an example.
With this last scheme, one can still find the loosely or explicit approaches and the strongly coupled or implicit approaches as alternatives.
Each of these strategies has its benefits and drawbacks, with the particular choice of one over the others depending on the particulars of the development environment.

\subsection{Semi-crystalline polymers}

Semi-crystalline polymers have a complex and hierarchial heterogeneous morphology.
Both their microstructure and their mesostructure will depend on the processing history, as well as mechanical and thermal histories, in addition to the polymer chemistry and conformation \citep{khouryMorphologyCrystallineSynthetic1976,cangemiTwoPhaseModelMechanical2001,hoffmanAnalysisRelaxationsPolychlorotrifluoroethylene2007}.

\paragraph{Microstructure}
At the microscopic level, they consist of at least two different phases: a crystalline phase and an amorphous phase \citep{khouryMorphologyCrystallineSynthetic1976}.
The ordered structure that composes the crystalline portion of a semi-crystalline polymer results from the constituent chains packing parallel to one another in an orderly fashion into lamellae, as show in Figure~\eqref{fig:crystalline_phase_of_scp}.

\begin{figure}[htbp]
	\includegraphics[width=0.9\textwidth]{example-image-a}
	\caption{Arrangement of polymer chains into a lamella in the crystalline phase of a semi-crystalline polymer.}
\label{fig:crystalline_phase_of_scp}
\end{figure}

There are at least two types of crystal lamellae found in semi-crystalline polymers, as detailed in \cite{andersonMorphologyIsothermallyBulk1964} for polyethylene (PE), the  chain-folded lamellae and extended-chain lamellae (see Figure~\eqref{fig:types_of_crystall}).
In the former, the molecular chains within each platelet fold back and forth on themselves, with folds occurring at the faces.
This is in fact, an idealization, with reality resembling more a switchboard model, with the chains reentering through loose folds at non-adjacent sites or even forming tie-chains with a neighboring lamellae \citep{gsellEvolutionMicrostructureSemicrystalline1994}.
The latter is more common at lower molecular weights with the chains organized into lamellae in their extended conformation.
The thickness of the lamellae in semi-crystalline polymers is of the order of nanometers, e.g., between \SIrange{10}{15}{\nano\meter} for PE samples \citep{argonPhysicsDeformationFracture2013a}.
\begin{figure}[hbtp]
	\includegraphics{example-image-a}
	\caption{Schematic depictions of chain-folded and extended-chain lamellae.}
\label{fig:types_of_crystall}
\end{figure}

Regarding the crystalline structure, it will depend on the polymer in question.
PE possesses most often an orthorhombic symmetry, where the chain direction forms an angle with the normal vector to the crystalline lamella ranging between 17 and $40^\circ$ \citep{nikolovMicroMacroConstitutive2000}.
On the other hand, the crystal structure found in polytetrafluoroethylene (PTFE) at temperatures above \SI{19}{\celsius} is hexagonal, with individual molecules arranged in helical conformations \citep{bergstromMechanicsSolidPolymers2015}.
See Figure~\ref{fig:crystalline_structure_PE_PTFE} for a depiction of both crystalline structures.
\begin{figure}[hbtp]
	\includegraphics[width=0.9\textwidth]{example-image-a}
	\caption{Depiction of the crystalline structures of polyethylene (PE) and polytetrafluoroethylene (PTFE).}
\label{fig:crystalline_structure_PE_PTFE}
\end{figure}

The crystallinity of a semi-crystalline polymer can be specified by the degree of crystallinity.
It may range from completely amorphous to almost entirely crystalline.
\cite{wardIntroductionMechanicalProperties2004} mentions values between $90\%$ for polyethylene (PE) to about $30\%$ for oriented poly(ethylene terephthalate) (PET).
Commercially available semi-crystalline polymers range from $10\%$ to $90\%$ in degree of crystallinity \citep{vandommelenMicromechanicalModelingElastoviscoplastic2003}.

The degree of crystallinity by weight may be determined from accurate density measurements, according to
\begin{equation}
	\chi = \% \text { crystallinity }=\frac{\left(\rho_{s}-\rho_{a}\right)/\rho_{s}}{\left(\rho_{c}-\rho_{a}\right)/\rho_{c}} \times 100
\end{equation}
where $\rho_{s}$ is the density of a specimen for which the percent crystallinity is to be determined, $\rho_{a}$ is the density of the totally amorphous polymer, and $\rho_{c}$ is the density of the perfectly polymer crystallite.
The values of $\rho_{a}$ and $\rho_{c}$ must be measured by other experimental means.
Other experimental methods employed to determine the crystallinity along with the lamellar thickness of the polymer crystallites include wide (WAXS) and small (SAXS) angle X-ray scattering \citep{schrauwenIntrinsicDeformationBehavior2004, hobeikaTemperatureStrainRate2000}, as well as, electron microscopic, e.g., transmission electron microscopy (TEM) \citep{bartczakEvolutionCrystallineTexture1992}.

The parameters influencing the crystallinity are mainly the molecular structure, the molecular weight, the presence of plasticizers, and especially the thermo-mechanical history of the polymer \citep{khouryMorphologyCrystallineSynthetic1976, cangemiTwoPhaseModelMechanical2001}.
Given the way that polymer crystals form, polymer chains must possess a linear structure.
The more branches/pendant side groups the lesser the degree of crystallinity.
Even linear polymers must, however, have sufficient regularity in order to crystallize \citep{khouryMorphologyCrystallineSynthetic1976}.
A high molecular weight tends to suppress a high degree of crystallinity \citep{hoffmanAnalysisRelaxationsPolychlorotrifluoroethylene2007}, as seen comparing high density polyethylene (HDPE) and ultra high weight polyethylene (UHWPE) \citep{brownInfluenceMolecularConformation2007}.
Given that the crystallization is a kinetic process the rate of crystallization in polymers is dependent on the temperature with larger temperatures leading to smaller rates \citep{callister2014materials}.
As detailed later in this chapter, the mechanical loading of a polymer may also lead to changes in its crystallinity, e.g., through the phenomenon of strain-induced crystallization \citep{raoStudyStraininducedCrystallization2001}.

Accordingly, the most frequent approach to achieve different degrees of crystallinity is through the control of crystallization temperatures and/or crystallization times, be it when crystallizing from the melt or through annealing treatments \citep{fakirovGlassTransitionTemperature2000, schrauwenIntrinsicDeformationBehavior2004}.
However, the preparation of samples with different degrees of crystallinity is not a routine task for polymers such as HDPE, since its rate of crystallization is very high.
One solution is to take PE samples differing in the degree of branching, since by introducing various amounts of defects in the main chain, it is possible to control the degree of crystallinity \citep{fakirovGlassTransitionTemperature2000}.

In what pertains to the amorphous portion of a semi-crystalline polymer, results reported by Zia et al. \citep{ziaRigidAmorphousFraction2008} on isotactic polypropylene (iPP), for example, point to the existence of two different amorphous phases, a mobile amorphous phase and rigid amorphous phase, on the basis of different glass transition temperatures.
The results of Jolly \citep{jollyAnalyseMicrostructurePolyamide2000} concerning polyamide 11 (PA11) found employing WASX, carried out at different axial deformation rates also support the existence of a neither pure amorphous nor crystalline phase.
According to Mandelkern \citep{mandelkernCrystallinePolymerReminiscences2006}, the existence of a rigid amorphous phase is supported by experimental experimental results obtained from density measurements, wide and small-angle X-ray diffraction, thermal analyses, Raman spectroscopy, small-angle neutron scattering, dielectric relaxation and nuclear magnetic resonance involving different nuclei and techniques.

The reason for the increased rigidity in this part of the amorphous phase is the presence of polymer crystallites, which hinder the molecular mobility of the amorphous phase \citep{ziaRigidAmorphousFraction2008, peacockHandbookPolyethyleneStructures2014}.
However, an increase in degree of crystallinity will lead to a decrease in the rigid amorphous fraction as well as the ratio between rigid and mobile amorphous phase.
This behavior is due to reduced covalent coupling between the polymer crystals and the amorphous phase in highly crystalline preparations\citep{ziaRigidAmorphousFraction2008}.
Furthermore, the presence of the crystallites also affects the properties of the mobile amorphous fraction, which is detected by a distinct decrease its glass transition temperature, as shown for semi-crystalline iPP by Zia et al. \citep{ziaRigidAmorphousFraction2008}.

\paragraph{Mesostructure}
According to the processing, thermal and mechanical history, as well as, its degree of crystallinity, molecular weight and polydispersity, a semi-crystalline polymer can display different mesoscopic structures \citep{cangemiTwoPhaseModelMechanical2001, mandelkernCrystallinePolymerReminiscences2006}.

When the polymer crystallizes from the a dilute solution the structure achieved is often lamellar and composed of multiple layers, if the solution is quiescent, and of the shish-kebab variety\footnote{The so-called shish-kebab structure consists of long central fiber core (shish) surrounded by lamellar crystalline structure (kebab) periodically attached along the shish.
\citep{naViscousForceDominatedTensileDeformation2006, peacockHandbookPolyethyleneStructures2014}.}, if the solution is subject to high shear \citep{khouryMorphologyCrystallineSynthetic1976, callister2014materials, peacockHandbookPolyethyleneStructures2014}.
When the polymer crystallizes from the melt, the two most commonly reported types of mesoscopic structures for semi-crystalline polymers are the spherulitic structure \citep{zengConstitutiveModelSemicrystalline2010}, obtained from quiescent crystallization, and the shish-kebab structure, obtained from crystallization under shear stress.
For a schematic depiction of a spherulitic and a shish-kebab structure see Figure~\ref{fig:mesostructure_scp}.
\begin{figure}[hbtp]
	\includegraphics[width=.9\textwidth]{example-image-a}
	\caption{Schematic depiction of a spherulitic and a shish-kebab mesostructure.}
\label{fig:mesostructure_scp}
\end{figure}

The spherulitic structure is composed of spherulites, an aggregate of ribbon-like chain-folded crystallites approximately 10 to \SI{20}{\nano\meter} thick for PE and 2 to \SI{6}{\nano\meter} for polyether ether ketone (PEEK), e.g., that radiate outward from a single nucleation site in the center, their diameter approximately \SI{10}{\micro\meter}.
Between them there are amorphous regions, crossed by tie-chain molecules that act as connecting links between adjacent lamellae \citep{callister2014materials, khouryMorphologyCrystallineSynthetic1976, pouriayevaliConstitutiveDescriptionRatesensitive2013, gsellEvolutionMicrostructureSemicrystalline1994}.
The lamellae are generally twisted about their long axis \citep{patlazhanStructuralMechanicsSemicrystalline2012}.
A sheave-like structure is also possible under suitable conditions \citep{peacockHandbookPolyethyleneStructures2014}.
Mandelkern \citep{mandelkernCrystallinePolymerReminiscences2006} warns however that spherulites, and other type of supermolecular structures, are not universally observed in homopolymers.

Mechanical loading will also lead to changes in the mesoscopic structure of the polymer.
Regarding higher crystallinity polymers such as HDPE, it normally results in the destruction of the crystallites of the original morphology, followed by reordering to form new crystallites.
The new lamellar morphology has a lamellar thickness independent of the original lamellar thickness, being solely dependent upon the temperature at which the deformation occurred .
In such morphologies the unit cell axes are preferentially aligned in the stress direction, while the lateral planes of the lamellae lie approximately normal to the aligning force.
These, newly formed crystallites are themselves subject to disruption at higher deformation levels (draw ratios of approximately 10), being replaced by a fibrillar morphology, which consists of oriented crystallites arranged hierarchically into needle-like structures of various sizes.
These macrofibrils are composed of microfibrils, in turn made up of nanofibrils, stacks of crystallites separated by thin noncrystalline “plates," portions of which are spanned by “intercrystalline bridges" \citep{peacockHandbookPolyethyleneStructures2014}.
A nearly perfect alignment of the crystallized chains along the fiber axis, as well as the parallel arrangement of the crystal lamellae relative to the same axis, define the final fiber structure \citep{peterlinMolecularModelDrawing1971}.
Moreover, plastically deforming HDPE develops three important types of texture, resembling that of a large quasi-single crystal \citep{argonPhysicsDeformationFracture2013a}:
\begin{enumerate}
	\item crystallographic texture due to preferential orientation of crystallographic axes in the crystalline lamellae;
	\item morphological texture due to preferential orientation of the normals to the broad faces of the crystalline lamellae faces; and
	\item macromolecular texture in the amorphous component, which is promoted by alignment of molecules with the direction of maximum stretch.
\end{enumerate}

At the other extreme, for materials such as PET, in which the crystalline and amorphous components are intermixed, the most noticeable effect may be strain-induced crystallization due to macromolecular texture as described above \citep{wardIntroductionMechanicalProperties2004}.
More specifically, temperatures above the glass transition temperature, a linear structure and large deformations lead to an increase in the crystallinity of the material \citep{ahziModelingDeformationBehavior2003}.
The crystalline structures produced in this way are oriented, which results in an anisotropic mechanical response.
Experiments on polymer film also point to crystallization at lower strain when the strain rate is higher \citep{raoStudyStraininducedCrystallization2001}.
In fact,  a majority of plastic products are manufactured by deforming the material at elevated temperatures to get it into the desired shape.
Common examples of these types of operations include film blowing, fiber spinning and injection molding.
In many of these applications, the formation of a highly oriented crystalline phase has a beneficial impact on the mechanical behavior of the material \citep{dairaniehPhenomenologicalModelFlowInduced1999, raoStudyStraininducedCrystallization2001}.
Most PET articles are manufactured in this way \citep{boyceConstitutiveModelFinite2000, raoStudyStraininducedCrystallization2001, makradiTwophaseSelfconsistentModel2005}.

\paragraph{Examples and applications}


\section{Computational Framework}

All the numerical simulations based on the Finite Element Method (FEM) are held in the in-house Fortran (IBM Mathematical Formula Translation System) program LINKS (Large Strain Implicit Non-linear Analysis of Solids Linking Scales), a multi-scale finite element code for implicit infinitesimal and finite strain analyses of hyperelastic and elastoplastic solids, that is continuously developed by the CM2S research group (Computational Multi-Scale Modeling of Solids and Structures) at the Faculty of Engineering of University of Porto.

In the present work, the author contributes to the addition of a suitable coupling environment for the partitioned solution of coupled fields and a thermal solver based on the Finite Elements Method.
Appropriate nonlinear solvers are also added as implicit solution strategies for the coupled thermomechanical problem.

\section{Objectives}

The main goals of this work are:
\begin{itemize}
    \item To describe in a thermodynamically consistent way the thermomechanical problem;
    \item To develop and validate a thermal solver based on the Finite Element Method;
    \item To provide a thorough overview of the available methods for the solution of coupled problems, in particular, the thermomechanical problem;
    \item To validate the thermomechanical solver and compare the strongly coupled partitioned strategies available in the literature.
\end{itemize}

\section{Document structure}

The remainder of this document is structured as follows:

\paragraph{Chapter \ref{ch:continuum_mechanics} - Continuum Thermomechanics}\mbox{}\\
This chapter covers the notions required to explain how a solid responds to thermal and mechanical loads under large deformations, including the conservation laws that guarantee mechanical equilibrium and energy conservation.
Additionally, the application of thermodynamics with internal variables is discussed, along with the resulting inferences about the constitutive behavior of the material that makes up the solid.

\paragraph{Chapter \ref{ch:mechanical_problem} - Mechanical problem}\mbox{}\\
This chapter presents the strictly mechanical problem, including the constitutive initial value problem, the weak form of the momentum balance equations, and the corresponding mechanical initial boundary value problem.
A brief description of the application of the Finite Element Method to this problem is also included.

\paragraph{Chapter \ref{ch:thermal_problem} - Thermal problem}\mbox{}\\
This chapter presents the strictly thermal problem, including the constitutive law for the heat flux, the weak form of the energy conservation equation, and the corresponding thermal initial boundary value problem.
A brief description of the application of the Finite Element Method to this problem is also included.

\paragraph{Chapter \ref{ch:thermo_mechanical_problem} - Thermomechanical problem}\mbox{}\\
This chapter presents the thermomechanical problem, including the constitutive initial value thermomechanical problem, the weak form of the energy conservation equation and the momentum balance equations, and the corresponding thermomechanical initial boundary value problem.
A brief description of the application of the Finite Element Method to this problem is also included.

\paragraph{Chapter \ref{ch:val_therm_solver} - Validation results for the thermal solver}\mbox{}\\
This chapter details the validation results for the thermal solver using as references the \cite{DINEN1991_1_2} and \cite{NAFEMSbenchmarks}.
It includes both transient effects and boundary conditions such as natural convection and radiation.

\paragraph{Chapter \ref{ch:sol_proc_coupl_fields} - Solution procedures for coupled fields}\mbox{}\\
This chapter presents an overview of the solution procedures for coupled problems.
It includes monolithic schemes and partitioned schemes, both explicit and implicit approaches.
An evaluation and discussion of the different methods are provided.

\paragraph{Chapter \ref{chapter:implicit_meth} - Implicit solution methods for coupled fields}\mbox{}\\
This chapter provides a thorough description of the available implicit methods.
It rests on recasting the problem as a simple root-finding problem for a set of nonlinear equations.
The methods presented are the fixed-point method, the underrelaxation method, the Aitken relaxation, the Broyden-like family of methods, the Newton-Krylov methods, and the polynomial vector extrapolation methods in cycling mode.
The number of residual evaluations, the memory requirements, the computational complexity, and the ease of implementation are all discussed for each approach.

\paragraph{Chapter \ref{ch:val_acc_techniques} - Numerical results for the implicit coupling schemes}\mbox{} \\
This chapter covers the validation results for the thermomechanical solver and the implicit schemes explored in this work.
Each class of implicit methods is examined for efficiency, including as a function of coupling strength, and the best methods in each category are contrasted.
The use of polynomial predictors is also explored.

\paragraph{Chapter \ref{ch:conclusions} - Conclusion and Future Works}\mbox{} \\
This chapter presents the conclusions reached in this work, and some future research directions are suggested.

\newpage\null\thispagestyle{blank}\newpage
