\chapter{Numerical results} \label{sec:numerical_results}

This chapter presents two sets of results concerning the subject of thermomechanical modeling of semi-crystalline polymers.
The first is adapted from \cite{vilacha} and pertains to the implicit partitioned solution of the thermomechanical model employing a elasto-plastic description for the material.
The nonlinear solvers available are compared and analysed according to their CPU time, memory use and ...
The second pertains to the implementation and validation of state of the art constitutive models for semi-crystalline polymers.
Two models are considered.
One is applicable to polymer plastics, in general, and does not concider the bulk crystallinity of the polymer excplicitly.
The other considers the crystallinity of the polymer explicitly and attempts to describe the polymer at different cyrstallinities with only one set of material parameters.


\section{Solution of the thermomechanical problem}


\section{State of the art in semi-crystalline polymer modeling}

This section presents results regarding the state of the art in semi-crystalline polymer modeling.
The first model is due to Hao et al. \citep{hao} and it does not consider cyrstallinity.
It

The reference experimental results were obtained by Khan and Farrohk \citep{khan} for Nylon101 in compression.



% \section{Wishlist}
%
% Independent variables
% \begin{itemize}
%   \item strain rate, temperature
%   \item stress, temperature
% \end{itemize}
%
% Loading type
% \begin{itemize}
%   \item uniaxial traction
%   \item uniaxial compression
%   \item pure shear
%   \item biaxial traction
%   \item hydrostatic compression
%   \item torsion
% \end{itemize}
%
% time dependency
% \begin{itemize}
%   \item impulse
%   \item constant
%   \item ramp
%   \item harmonic
%   \item cyclic
% \end{itemize}
%
% The most interesting loading schemes are:
% \begin{itemize}
%   \item constant strain rate;
%   \item stress relaxation
%   \item creep
%   \item jump constant strain rate at different strains
%   \item loading unloading
%   \item free shrinkage
% \end{itemize}
%
%
% \section{Constant strain rate experiments}
%
% The experiments employed will be.
%
% Compare yield stresses.
%
% Verify thermal sofetening.
%
% Try and choose polymers above and below the glass transition temperature. Need to model the intrinsic softening.
%
% Check how the initial stiffness varies with the strain rate and the temperature.
%
% \section{Unloading}
%
% Remaining deformation Strobl and friends
%
% \section{Free shrinkage}
%
% Remaining deformation
%
% \section{Heated shrinkage}
%
% \colorbox{CyanBlue}{Try the self healing stuff for fibers.}
%
% \section{Stress relaxation}
%
% Test with stress relaxation, maybe master curve.
%
% \section{Creep}
%
% Test with creep, maybe master curve.
