\chapter{Continuum thermomechanics} \label{ch:continuum_mechanics}

This chapter covers the notions required to explain how a solid responds to thermal and mechanical loads under large deformations, including the conservation laws that guarantee mechanical equilibrium and energy conservation.
Additionally, the application of thermodynamics with internal variables is discussed along with the resulting inferences about the constitutive behavior of the material that makes up the solid.
These topics are broadly covered in the literature and here the approach used follows \cite{de_souza_neto_computational_2008},

\section{Kinematics of Deformation}

\subsection{Motion}

Let a deformable body $\mathscr{B}$ occupy an open region $\Omega_0$ of the tridimensional Euclidean space $\mathscr{E}$ with a regular boundary $\partial \Omega_0$ in its reference configuration.
Its motion, depicted in Figure \ref{fig:motion}, is defined by a smooth one-to-one function
\begin{equation}
    \vect \varphi\colon \Omega\times \mathscr{R}\to \mathscr{E},
\end{equation}
mapping each material particle of coordinates $\vect X$ in the reference configuration to its position $\vect x$ in the deformed configuration, for a given instant of time $t$, as
\begin{highlight}
    \begin{equation}
        \vect x = \vect \varphi(\vect X, t)=\bm \varphi_t(\bm X).
    \end{equation}
\end{highlight}

\enlargethispage{\baselineskip}
Thus, the displacement field is defined as
\begin{equation}
    \vect u(\vect X, t) = \vect \varphi(\vect X, t) - \vect X,
\end{equation}
and, since the function that defines the motion is one-to-one, the reference configuration can be recovered as
\begin{equation}
    \vect X = \vect \varphi^{-1} (\vect x, t) = \vect x - \vect u(\vect \varphi^{-1}(\vect x, t),t),
\end{equation}
where $\vect \varphi^{-1}$ is the reference mapping function.

\begin{figure}
  \centering
  % \includegraphics[width=.9\textwidth]{motion}
  \caption{Motion}
\label{fig:motion}
\end{figure}

\subsection{Material and spatial descriptions}

Dealing with finite deformations, the behavior of the body under analysis can be described with respect to the reference configuration, using the so-called material or Lagrangian description, or to the deformed configuration, using the so-called spatial or Eulerian description.

In the Lagrangian description any field, be it scalar, vectorial or tensorial defined over the body is expressed as a function of the reference configuration, $\vect X \in \Omega_0$.
On the other hand, the Eulerian descritpion of same field is done using the deformed configuration, $\vect x\in \Omega$.

As such let $\alpha(\vect x,t)$ be a spatial field and $\beta(\vect X, t)$ a material field.
Their material $\alpha_m$ and spatial $\beta_s$ desctiptions are given by
\begin{align}
    \alpha_m(\vect X, t) &=\alpha(\vect \varphi(\vect X, t),t),\\
    \beta_s(\vect x, t) &= \beta(\vect \varphi^{-1}(\vect x, t),t),
\end{align}
noting that any field associated with a motion of $\mathscr{B}$ can be expressed as a function of time and the point's position in the reference or deformed configuration.

The same distinction between material and spatial descriptions applies to operators such as the divergence and the gradient.
The spatial and material gradients, $\nabla$ and $\nabla_0$, respectively, are defined as
\begin{equation}
    \nabla \alpha = \frac{\partial}{\partial \vect x}\alpha(\vect x,t),\quad
    \nabla_0 \beta = \frac{\partial}{\partial \vect X}\beta (\vect X, t),
\end{equation}
where the derivatives are taken with respect to the spatial and reference configuration accordingly.

% \subsection{Velocity and acceleration}
%
% For solid dynamical problems, both the velocity and the acceleration corresponding to the motion \(\bm \varphi\) must be considered.
% The velocity is defined as
% \begin{gather}
%   \bm V = \frac{\partial \bm \varphi}{\partial t},\\
%   \bm v_t = \bm V_t \circ \bm \varphi_t^{-1},
% \end{gather}
% in the material and and spatial descriptions, respectively.
% See Figure~\ref{}.
% The formulas for the acceleration, again in both material and spatial descriptions, are
% \begin{gather}
%  \bm A = \frac{\partial \bm V}{\partial t},\\
%  \bm a = \bm A_t\circ \bm \varphi^{-1}_t.
% \end{gather}
% The spatial acceleration \(\bm a\) is related to the spatial velocity \(\bm v\) by
% \begin{equation}
%   \bm a = \dot{\bm v} = \frac{\partial \bm v}{\partial t} + \nabla \bm v\cdot \bm v.
% \end{equation}


\subsection{Deformation gradient}

The deformation gradient, a second order tensor denoted by $\mat F$, is defined as
\begin{highlight}
    \begin{equation}
            \mat F(\vect X,t)\equiv \nabla_0\vect\varphi(\vect X,t) =\frac{\partial \vect x}{\partial \vect X},
    \end{equation}
\end{highlight}
or, taking into account that
\begin{equation}
    \vect x = \vect X + \vect u(\vect X, t),
\end{equation}
it can be expressed as
\begin{equation}
    \mat F(\vect X, t) = \mat I + \nabla_0 \bm u.
\end{equation}

The deformation gradient relates the relative position between two neighboring material particles before and after deformation.
To see this let $\vect X$ be the coordinates of some material particle in the reference configuration and $\vect X + d\vect X$ the coordinates of some material particle in its neighborhood, their corresponding coordinates in the deformed configuration are given by
\begin{gather}
    \vect X = \vect x - \vect u(\vect X, t), \label{eq:mat_pos}\\
    \vect X + d\vect X = \vect x + d\vect x -\vect u(\vect X +d\vect X, t). \label{eq:mat_pos_delta}
\end{gather}
Subtracting Equation \eqref{eq:mat_pos} to Equation \eqref{eq:mat_pos_delta}, it is found that
\begin{align}
    d\vect X &= d\vect x +\vect u(\vect X, t)-\vect u(\vect X +d\vect X, t)\\
             &=(\mat I +\nabla_0 \vect u(\vect X, t))\; d\vect x\\
             &=\mat F\;d\vect x.
\end{align}

Due to this relation, it can be shown that the determinat of the deformation gradient has a physical meaning.
It is the local unit volume change, that is,
\begin{highlight}
\begin{equation}
   J\equiv \text{det}\;\mat F = \frac{dv}{dv_0}, \label{eq:det_grad_func}
\end{equation}
\end{highlight}
where $dv_0$ is an infinitesimal volume of the body in its reference configuration and $dv$ the infinitesimal volume after deformation.

\subsubsection{Isochoric/Volumetric decomposition}

Any deformation can be locally decomposed in volumetric and isochoric (or distortional) components.
From Equation \eqref{eq:det_grad_func} it can be gathered that an isochoric deformation is characterized by $J=1$.
As such, the deformation gradient can be decomposed as
\begin{equation}
    \mat F = \mat F_\text{iso} \mat F_\text{vol} = \mat F_\text{vol} \mat F_\text{iso},
\end{equation}
where the isochoric and columetric components are defined by
\begin{equation}
    \mat F_\text{iso} = (\text{det}\;\mat F)^{-\frac{1}{3}},\quad \mat F_\text{vol} = (\text{det}\;\mat F)^{\frac{1}{3}}\mat I.
\end{equation}

\subsubsection{Polar decomposition}

The deformation gradient can also be decomposed in rotation and stretch components, the so-called polar decomposition, defined as
\begin{highlight}
    \begin{equation}
        \mat F = \mat R\mat U = \mat V\mat R, \label{eq:polar_decomposition}
    \end{equation}
\end{highlight}
where $\mat R$ is the proper orthogonal rotation tensor and $\mat U$ and $\mat V$ are the symmetric positive right and left stretch tensors, respectively.

Equation \eqref{eq:polar_decomposition} has a physical interpretation with the right polar decomposition ($\mat F= \mat R\mat U$) corresponding to a stretch mapping followed by a rotation, and the left polar decomposition ($\mat F = \mat V\mat R$) corresponding to a rotation followed by a stretch mapping.
\
The right $\mat U$ and left $\mat V$ stretch tensors are related through the rotation matrix $\mat R$ as
\begin{equation}
    \mat V = \mat R\mat U\mat R^T,
\end{equation}
and can be obtained from deformation gradient by
\begin{highlight}
    \begin{equation}
        \mat C \equiv \mat U^2 = \mat F^T \mat F,\quad \mat B \equiv \mat V^2 = \mat F\mat F^T.,
    \end{equation}
\end{highlight}
where $\mat C$ and $\mat B$ are the right and left Cauchy-Green strain tensors.

Since $\mat U$ and $\mat V$ are symmetric tensors, they admit the spectral decomposition
\begin{equation}
    \mat U = \sum_{i=1}^3 \lambda_i \vect E_i^* \otimes \vect E_i^*,\quad \mat V = \sum_{i=1}^3 \lambda_i \vect e_i^*\otimes \vect e_i^*,
\end{equation}
where $\lambda_i$, $i=1,2,3,$ are the eigenvalues of both $\mat U$ and $\mat V$ and $\vect E_i^*$ and $\vect e_i^*$ are the respective eigenvectors.

The eigenvectors of left $\mat V$ and right $\mat U$ stretch tensors are related through
\begin{equation}
    \vect e_i^*= \mat R \vect E_i^*.
\end{equation}
forming two orthogonal bases.
These vectors define the Lagrangian and Eulerian principal directions, respectively, allowing for the expression of the local stretching from a material particle, associated with any deformation, as a superposition of stretches along the three mutual orthogonal directions.,

% \textcolor{red}{
% For solid dynamical problems, material time differentiation of the deformation and the strain measures need to be introduced. The first and second derivative of the displacement field \(u(X, t)\). li.e. the material velocity and acceleration, \(\dot{u}\) and \(\ddot{u}\), respectively, result in)
% where \((2.2)\) can be used to show that due to the constant initial position \(X\) )
% \[
% \hat{\boldsymbol{x}}(\boldsymbol{X}, t) \equiv \overline{\boldsymbol{u}}(\boldsymbol{X}, t)
% \]
% According to \((2.3)\), the material velocity gradient results in
% With the material gradient operator \(\operatorname{Grad}(\cdot)\). Equation \((2.23)\) can be reformulated, yielding the Ispatial velocity gradient
% and its symmetric part
% \[
% \begin{array}{c}
% \boldsymbol{L}=\dot{\boldsymbol{F}} \cdot \boldsymbol{F}^{-1} \\
% \boldsymbol{D}=\frac{1}{2}\left(\boldsymbol{L}+\boldsymbol{L}^{\top}\right)
% \end{array}
% \]
% The rate of the Green-Lagrange strain tensor results in the material strain rate tensor
% \[
% \dot{E}_{\mathrm{GL}}=\frac{1}{2} \dot{C}=\boldsymbol{F}^{\mathrm{T}} \cdot \boldsymbol{D} \cdot \boldsymbol{F}
% \]
% which can be expressed as pull-back of the symmetric spatial strain rate tensor \(D\). Via the Liederivative or Oldroyd-Lie derivative of the Euler-Almansi strain tensor \(\mathcal{L}_{t}\left[\boldsymbol{E}_{\mathrm{EA}}\right]\), which represents an objective material time derivative, a relation to the rate of the Euler-Almansi strain tensor is achieved as
% \[
% \mathcal{L}_{t}\left[\boldsymbol{E}_{\mathrm{EA}}\right]=\varphi\left[\frac{\mathrm{d}}{\mathrm{d} t}\left(\varphi^{-1}\left[\boldsymbol{E}_{\mathrm{EA}}\right]\right)\right]=\boldsymbol{D}=\boldsymbol{F}^{-\top} \cdot \dot{\boldsymbol{E}}_{\mathrm{GL}} \cdot \boldsymbol{F}^{-1}
% \]
% Moreover, the rate of volume changes is described by \(\dot{J}\) and can be expressed through
% \[
% \dot{J}=J \operatorname{tr} \boldsymbol{D}
% \]}


\section{Strain tensors}

In Continuum Mechanics there are two main families of strain tensors derived from the deformation gradient and used to describe the body deformation.
The Lagrange family strain tensors are defined as
\begin{equation}
    \mat E^{(m)} =\begin{cases} \displaystyle{\frac{1}{m} (\mat U^m - \mat I)},&\quad m\neq 0,\\[12pt] \ln(\mat U),& \quad m=0, \end{cases}
\end{equation} where $m$ is a real number, and likewise, the Euler family strain tensors are defined as
\begin{equation}
     \mat \varepsilon^{(m)} =\begin{cases} \displaystyle{\frac{1}{m} (\mat V^m - \mat I)},&\quad m\neq 0,\\[12pt] \ln(\mat V),& \quad m=0, \end{cases}
\end{equation}
where $m$ is also real number.

\enlargethispage{\baselineskip}
In particular, choosing $m=0$, one obtains the so-called material and spatial logarithmic strain tensors
\begin{highlight}[innertopmargin=-5pt]
    \begin{align}
        \mat E^{(0)} &\equiv \ln[\mat U] = \sum_{i=1}^3 \ln \lambda_i \vect E_i^*\otimes \vect E_i^*,\\
        \mat e^{(0)} &\equiv \ln [\mat V] = \sum_{i=1}^3 \ln \lambda_i \vect e_i^*\otimes \vect e_i^*.
    \end{align}
\end{highlight}

\colorbox{BrickRed}{Explain somewhere what is the true strain.}

\section{Forces and stress measures}

The deformation of a body is intrinsically related to the forces acting on it.
These forces can be divided in two classes, from a purely mechanical point of view: volume (or body) forces, proportional to the mass contained in a volume element, as such measured in force per unit volume, and surface forces, acting on the surface of a volume element, measured as force per unit area.
Related to the latter is the concept of stress, that can be described mathematically by second order tensors with different definitions.

\subsubsection{Cauchy stress tensor}

According to Cauchy's theorem the relation between the so-called Cauchy stress vector, $\vect t(\vect x,\vect n)$, and the unitary outward vector normal to the deformed surface under analysis, $\vect n$, is linear and given by
\begin{highlight}
    \begin{equation}
        \vect t(\vect x,\vect n)\equiv \mat \sigma(\vect x)\vect n,
    \end{equation}
\end{highlight}
where $\mat \sigma$ is the second order Cauchy stress tensor.

The Cauchy stress vector is naturally associated with the deformed configured and thus, expressed in a spatial description and measured in force per unit deformed area.
It must also be noted that as a consequence of the balance of angular momentum, the Cauchy stress tensor is symmetric.

\subsubsection{First Piola-Kirchhoff stress tensor}

    The First Piola-Kirchhoff stress tensor, $\mat P$, can be regarded as the material counterpart of the Cauchy stress tensor, as it establishes a linear dependence between the stress vector $\vect t_0(\vect X,\vect m)$, measured in force per unit reference area, and the unitary outward vector normal to the undeformed surface under analysis, $\vect m$,
    \begin{equation}
        \vect t_0 = \mat P \vect m,
    \end{equation}
    which must related to the Cauchy stress vector by
    \begin{equation}
        \vect t_0 =\frac{\ud a}{\ud a_0} \vect t = \frac{\ud a}{\ud a_0} \mat \sigma \vect n,
    \end{equation}
    where $\ud a$ is the infinitesimal deformed area normal to the unitary vector $\vect n$ and $\ud a_0$ the corresponding undeformed are normal to $\vect m$.
    It can be shown that the relation between $da$ and $da_0$ is
    \begin{equation}
        \frac{da}{da_0}\vect n=J\mat F^{-T}\vect m,
    \end{equation}
    and substituting on the equation above motivates the following definition
    \begin{highlight}
        \begin{equation}
            \mat P \equiv J\mat \sigma \mat F^{-T}, \label{eq:def_piola}
        \end{equation}
    \end{highlight}
     where $J$ is the determinant of the deformation gradient $\mat F$ and $\mat \sigma$ is the Cauchy stress tensor.
     From Equation \eqref{eq:def_piola}, one gathers that, in general, the First Piola-Kirchhoff stress tensor is not symmetric.


\subsubsection{Kirchhoff stress tensor}

The Kirchhoff stress tensor, $\mat \tau$, is a widely used symmetric tensor, defined as
\begin{highlight}
    \begin{equation}
        \mat \tau \equiv J\mat \sigma.
    \end{equation}
\end{highlight}

\subsubsection{Deviatoric/Hydrostatic decomposition}

The Cauchy stress tensor, $\mat \sigma$, can split as
\begin{highlight}
    \begin{equation}
        \mat \sigma = \mat \sigma_d - p\mat I,
    \end{equation}
\end{highlight}
where $p$ is the hydrostatic pressure defined as
\begin{equation}
    p \equiv -\frac{1}{3}\text{tr}\;[\mat \sigma],
\end{equation}
and $\mat \sigma_d$ is the deviatoric stress defined as
\begin{equation}
    \sigma_d \equiv \mat \sigma - p\mat I.
\end{equation}

\section{Heat}

Heat flowing inside a body, entering or leaving it, often leads to temperature changes.
In Continuum Thermomechanics, heat is measured in power per unit surface.

\subsubsection{Heat flux vector}

According to Cauchy's theorem the relation between the heat flux across a surface, \(h(\bm x, \bm n)\), and the unitary outward normal to the deformed surface under analysis, \(\bm n\), is linear and given by
\begin{highlight}
  \begin{equation}
    h(\bm x, \bm n ) = -\bm q(\bm x)\cdot \bm n.
  \end{equation}
\end{highlight}
where \(\bm q\) is the heat flux vector.

\section{Fundamental conservation principles} \label{sec:fundamental_conservation_princ}

In Continuum Mechanics. there is a set of conservation principles and thermodynamic laws, that irrespective of the quantities used to describe the mechanical behavior of a body undergoing large deformations must always be satisfied.

\subsection{Principle of mass conservation}

The principle of mass conservation can be stated as
\begin{highlight}
    \begin{equation}
        \dot \rho + \rho\; \text{div}\; \dot{\bm u}=0,
    \end{equation}
\end{highlight}
    where $\rho$ is the material density measured in mass per unit deformed volume.

\subsection{Principle of linear momentum conservation}

The principle of linear momentum conservation can be stated in both material and spatial description.
In a spatial description it reads
\begin{highlight}
    \begin{equation}
        \begin{cases}
            \text{div}\;\mat \sigma + \vect b = \rho \ddot{\bm u},&\quad \forall\vect x\in \Omega,\\[12pt]
            \vect t = \mat \sigma \vect n,&\quad \forall\vect x\in\partial\Omega,
        \end{cases} \label{eq:material_equilibrium}
    \end{equation}
\end{highlight}
where $\vect b$ is the body forces field measured as per unit deformed volume.

One can also write the principle of linear momentum conservation in material coordinates, as
\begin{equation}
    \begin{cases}
        \text{div}_0\;\mat P + \vect b_0 = \rho_0 \ddot{\bm u},&\quad\forall \vect X\in \Omega_0,\\[12pt]
        \vect t_0 = \mat P \vect m,&\quad \forall\vect X\in\partial\Omega_0,
    \end{cases}\label{eq:spatial_equilibrium}
\end{equation}
where $\vect b_0$ is the body forces field, measured in force per unit undeformed volume, and $\rho_0$ is the material density, measured in mass per unit undeformed volume.
Both these quantities can be found from their spatial counterparts as
\begin{equation}
    \vect b_0 = J\vect b,\quad \rho_0 = J\rho.
\end{equation}
Take notice of the abuse of language regarding functions defined on the referece configuration \(\Omega_0\) and on the deformed configuration \(\Omega\).
The same symbol, \(f\), is used for a function \(f\) defined on \(\Omega\) and the function \(f\circ \bm\varphi\) defined on \(\Omega_0\).

Equations \eqref{eq:material_equilibrium} and \eqref{eq:spatial_equilibrium} are the so-called strong, point-wise or local equilibrium equations, as they enforce the mechanical equilibrium at every material particle of the body.

\subsection{First principle of thermodynamics} \label{sec:first_principle_thermo}

Let \(e\) be the internal energy per unit mass, \(r\) the heat supply per unit mass and \(\bm q\) the heat flux, then the first principle of thermodynamics pertaining to the balance of energy can be written in the spatial description as
\begin{highlight}
\begin{equation} \label{eq:first_principle_thermo}
  \begin{cases}
    \rho\dot e = \mat \sigma\;:\;\mat D + \rho r -\text{div}\;\vect q,\quad &\forall \bm x\in\Omega,\\
    \bm t = \bm \sigma \bm n,\quad &\forall \bm x\in \partial \Omega,\\
    h = \bm q \cdot \bm n,\quad &\forall \bm x\in \partial \Omega.
  \end{cases}
\end{equation}
\end{highlight}
The second order tensor $\mat D$ denotes a strain rate measure, such that the double contraction $\mat \sigma\colon\mat D$ represents the stress power per unit volume in the deformed configuration of body.
In material coordinates, it reads
\begin{equation}
  \begin{cases}
 \rho_0 \dot e = \bm P :\dot{\bm F} + \rho_0 r -\operatorname{div}_0 \bm q_0,\quad& \forall \bm X\in\Omega_0,\\
 \bm t_0 = \bm P\bm m,\quad&\forall \bm X\in\partial \Omega_0,\\
 h_0 = \bm q_0\cdot\bm m,\quad&\forall \bm X\in\partial \Omega_0,
  \end{cases}
\end{equation}
where \(\bm q_0\) is the Piola transformation of \(\bm q\), i.e.,
\begin{equation}
  \bm q_0 = J \bm F^{-T} \bm q,
\end{equation}
and
\begin{equation}
  h_0 = J h.
\end{equation}


\subsection{Second principle of thermodynamics}

The local entropy balance can written as
\begin{equation}
  \rho \dot s = -\operatorname{div}\left[\frac{\bm q}{\theta}\right] + \frac{\rho r}{\theta} + \hat{s},
\end{equation}
where \(\hat{s}\) is the entropy production.
The second principle of thermodynamics postulates that the changes in the entropy in the universe can never be negative, which is mathematically expressed by
\begin{equation}
  \hat s \geq 0,
\end{equation}
yielding
\begin{equation}
    \rho\dot s +\text{div}\left[\frac{\vect q}{\theta}\right] -\frac{\rho r}{\theta} \geq 0,
\end{equation}
   where $\theta$ and $s$ are the temperature and specific entropy fields respectively.
In a material description, it reads
\begin{equation}
  \rho_0 \dot s + \operatorname{div}_0 \left[\frac{\bm q_0}{\theta}\right] - \frac{\rho_0 r}{\theta} \geq 0.
\end{equation}

\subsection{Clausius-Duhem inequality}

Combining the first and second thermodynamic principles yields
\begin{equation}
    \rho\dot s + \text{div}\;\left[\frac{\vect q}{\theta}\right] -\frac{1}{\theta}(\rho\dot e -\mat\sigma\;:\bm D\; +\text{div}\;\vect q)\geq 0,
\end{equation}

From the definition of the specific Helmholtz free energy
\begin{highlight}
\begin{equation} \label{eq:def_helmholtz_free_energy}
    \psi \equiv e -\theta s,
\end{equation}
\end{highlight}
and defining the temperature field gradient as $\bm g=\nabla \theta$, it is possible to establish the so-called Clausius-Duhem inequality in the spatial description as
\begin{highlight}
    \begin{equation} \label{eq:clasius_duhem_eq}
        \underbrace{\mat \sigma:\mat D - \rho\left(\dot \psi +s\dot \theta\right)}_{\pazocal D_\text{int}} \underbrace{-\frac{1}{\theta}\vect q\cdot \vect g }_{\pazocal D_\text{cond}}\geq 0,
    \end{equation}
\end{highlight}
where the identity
\begin{equation}
    \text{div}\;\left[\frac{\vect q}{\theta}\right] =\frac{1}{\theta}\text{div}\;\vect q -\frac{1}{\theta^2}\vect q\cdot \nabla \theta.
\end{equation}
is used.

From a physical point of view, the Clausius-Duhem inequality states that the energy dissipation per unit deformed volume is always non-negative.
Moreover the terms in the inequalitiy can be splited into the internal dissipation \(\pazocal D_\text{int}\) and the dissipation due to heat conduction \(\pazocal D_\text{cond}\).
From
\begin{equation}
\hat s = \bm \sigma :\bm D - \rho \left(\dot \psi  +s \dot \theta\right) -\frac{1}{\theta}\bm q\cdot\bm g,
\end{equation}
assuming that the process leads to an uniform temperature distribution, yields for the internal dissipation \(\pazocal D_\text{int}\),
\begin{equation} \label{eq:def_int_dissipation}
\pazocal D_\text{int} = \hat{s}|_{\text{$\theta$ uniform} }= \bm \sigma:\bm D - \rho \left(\dot \psi + s \dot \theta\right),
\end{equation}
since conduction is excluded.
If on the other hand, only conduction effects are retained, the dissipation due to conduction, \(\pazocal D_\text{cond}\), is obtained as
\begin{equation} \label{eq:dissipation_conduction}
\pazocal D_\text{cond} = - \frac{1}{\theta} \bm q\cdot \bm g.
\end{equation}

Equation \eqref{eq:clasius_duhem_eq} can also be written as
\begin{equation}
  \bm P\colon\dot{\bm F} - \rho_0(\dot\psi + s\dot \theta) - \frac{1}{\theta}\bm q_0\cdot \bm g_0 \geq 0,
\end{equation}
where \(\bm g_0 = \nabla_0 \theta\), aplying the Piola transformation, and as
\begin{equation}
    \mat \tau\;:\;\mat D -\rho_0\left(\dot\psi + s\dot \theta\right) -\frac{J}{\theta}\vect q\cdot \vect g \geq 0,
\end{equation}
multiplying it by $J$ and attending to the definition of the Kirchhoff stress tensor, where the left hand side represents now the energy dissipation per unit reference volume.

\colorbox{BrickRed}{Say something about the energy equation and the dissipation.}

\section{Constitutive theory} \label{sec:thermomechanical_constitutive}

In Continuum Mechanics, the goal is often to obtain the stress, body forces, internal energy, entropy, heat source, and heat flux as a function of time and position, i.e., the calorodynamic process of a body $\mathscr B$ from the history of the motion and temperature, the so-called thermokinetic process.
This must, however, be achieved while adhering to the fundamental conservation principles already described.
Taking stock of the number of available equations, two, i.e., the balance of momentum (Equation~\eqref{eq:material_equilibrium}) and the balance of energy (Equation~\eqref{eq:first_principle_thermo}), and the number of unknown quantities, six, shows that the problem is currently ill-posed with an excess of unknowns.
Assuming that the quantities obtained from the two equations are the body forces, $\mathbf b$, and the heat source, $\mathbf r$, constitutive relations must be defined to enable the computation of the remaining quantities.

The approach adopted in the present text to provide the relevant constitutive relations is rational thermodynamics as developed by Truesdell, Noll, Coleman and others \citep{nollMathematicalTheoryMechanical1958, colemanFoundationsLinearViscoelasticity1961, colemanExistenceCaloricEquations1964}.
It is a framework broad enough to describe thermodynamically irreversible processes, i.e., processes involving dissipation, thus, allowing for the modeling of physical phenomena relevant to engineering practice.
A thorough review of the different approaches to the thermodynamics of irreversible processes and criticism of the strategy adopted here can be found in \cite{lavenda1978thermodynamics}.

Rational thermodynamics is typically developed through a set of axioms, having at its core the use of the Clausius-Duhem inequality to determine restrictions on the constitutive laws to be defined.
Adopting the axiom of thermodynamic determinism entails that the history of the thermokinetic process to which a neighborhood of a point $\mathbf p$ of the body $\mathcal B$ has been subjected determines the calorodynamic process for $\mathcal B$ at $\mathbf p$.
Thus, following the treatmnet in \cite{marsden1994mathematical}, given that the balance of energy and the balance of momentum supply two quantities from the calorodynamic process, here assumed to be the specific heat source, $r$, and the body force, $\mathbf b$, there still must be four functionals, $\hat{\bm \Sigma}$, $\hat{\mathbf Q}$, $\hat \Psi$, $\hat S$, supplying the Cauchy stress tensor, $\bm \sigma$, the heat flux vector, $\mathbf q$, the specific free energy, $\psi$ and the specific entropy $s$, respectively.
Accordingly, the functionals are defined as
\begin{gather}
  \label{eq:functional_stress}
  \hat{\bm \Sigma}\colon \mathfrak{H} \to S_2(\Omega), \\
  \label{eq:functional_heat_flux}
  \hat{\mathbf Q}\colon \mathfrak{H} \to \mathfrak{X}(\Omega), \\
  \label{eq:functional_free_energy}
  \hat{\Psi}\colon \mathfrak{H} \to \mathscr{F}(\Omega), \\
  \label{eq:functional_entropy}
  \hat{S}\colon \mathfrak{H} \to \mathscr{F}(\Omega),
\end{gather}
where the set of past histories, $\mathfrak H$, is given by
\begin{equation}
  \mathfrak H =\bigcup_{t\in \mathbb R}(\mathfrak M_t\times \mathfrak J_t \times \{t\}),
\end{equation}
the set of all past motions up to time $t$ is
\begin{equation}
  \mathfrak M_t =\{\bm\varphi\colon \Omega_0 \times(-\infty, t]\to \mathscr E |\ \text{$\bm\varphi$ is a $C^r$ regular motion of $\mathcal B$ in $\mathscr E$ for $-\infty < \tau \leq t$}\}
\end{equation}
and the of all past temperature fields up to time $t$ is
\begin{equation}
  \mathfrak J_t=\{\theta\colon\Omega_0\times (-\infty, t] \to (0, \infty)\ |\ \text{$\theta$ is a $C^r$ function}\},
\end{equation}
with $r$ a positive integer or $+\infty$.
Also $S_2(\Omega)$ denotes the space of $C^s$ symmetric contravariant two-tensor fields on $\Omega$ (for some suitable positive integer $s$), $\mathfrak{X}$ denotes the $C^s$ vector fields on $\Omega$, and $\mathscr F$ denotes the $C^s$ scalar fields on $\Omega$.

The axiom of admissibility or entropy production is also of special importance, providing a way to restrict the form the constitutive relations take.
This is achieved assuming that the Clausius-Duhem inequality, as given in Equation~\eqref{eq:clasius_duhem_eq}, always holds.
Another fundamental constitutive axioms, such as the principle of material objectivity (or frame invariance), the axiom of locality and the principle of material symmetry, must also be respected.
\colorbox{BrickRed}{Take into account the criticism of \cite{edelenMaterialIndifferencePrinciple1973} regarding material indiference.}

This formulation of the constitutive equations is general enough to include rate effects, i.e., dependence on strain rates, and memory effects, i.e., dependence on past histories.
However, some restrictions must be enforced on the functionals presented in Equations~\eqref{eq:functional_stress} to \eqref{eq:functional_entropy} in order to make thermodynamic determinism usable in a computational setting.

A thermoelastic material is described by a set of constitutive relations which depend only on the current deformed configuration and temperature field.
As a result, both rate and memory effects are eliminated from the possible behaviors displayed by this class of materials.
Supposing that the axiom of locality and entropy production hold, then the free energy function $\psi$ depends only on the variables $X$, $\mathbf F$, and $\theta$, i.e., $\psi = \hat \Psi(\phi, \theta)$ is a function of only $X$, $\mathbf F$, and $\theta$.
The corresponding constitutive equations are as follows
\begin{highlight}[innertopmargin=-5pt]
    \begin{gather}
    		\label{eq:constitutive_equation_stress_thermoelasticity}
        \mat \sigma = \rho \frac{\partial \psi}{\partial \mat F}\mat F^T, \\
        \label{eq:constitutive_equation_entropy_thermoelasticity}
        s = -\frac{\partial \psi}{\partial \theta}.
    \end{gather}
\end{highlight}
As expected, the internal dissipation is zero, with the entropy production inequality reducing to
\begin{highlight}
  \begin{equation}
    \mathbf q\cdot \mathbf g \leq 0.
  \end{equation}
\end{highlight}
See \cite{marsden1994mathematical} for the full derivation.

To allow for some rate effects, one can postulate the dependency of the free energy, stress, entropy and heat flux on the time derivative of the deformation gradient in addition to the deformation gradient itself, i.e.,
\begin{highlight}[innertopmargin=-5pt]
  \begin{gather}
\psi = \psi(\mathbf F, \dot{\mathbf F}, \theta, \mathbf g),\\
\bm \sigma = \bm \sigma(\mathbf F, \dot{\mathbf F}, \theta, \mathbf g),\\
s = s(\mathbf F, \dot{\mathbf F}, \theta, \mathbf g),\\
\mathbf q = \mathbf q(\mathbf F, \dot{\mathbf F}, \theta, \mathbf g).
\end{gather}
\end{highlight}
Coleman and Mizel \citep{colemanExistenceCaloricEquations1964} considers this hypothesis and concludes that the free energy cannot depend on $\dot{\mathbf F}$.
The constitutive equation found for the entropy is the same as shown in Equation~\eqref{eq:constitutive_equation_entropy_thermoelasticity}, and thus the entropy will also not depend on the time derivative of the deformation gradient.
Regarding the stress, it is no longer completely specified by the free energy, with only its equilibrium portion being given by the derivative of the free energy with respect to the deformation gradient.
The constitutive equation found has the form
\begin{highlight}
  \begin{equation}
    \bm \sigma = \bm \sigma_E(\mathbf F, \theta) + \bm \sigma_D(\mathbf F, \dot{\mathbf F}, \theta, \mathbf g),
  \end{equation}
\end{highlight}
where the subscripts $E$ and $D$ stand for elastic and dissipative.
The elastic term is obtained according to Equation~\eqref{eq:constitutive_equation_stress_thermoelasticity}, and is a function of the deformation gradient and the temperature only, while the additional dissipative term can depend also on both the time derivative of the deformation gradient and the temperature gradient.
However, the Clausius-Duhem inequality limits this overstress as it must be ensured that the dissipation is not negative.
More specifically, the inequality demands that
\begin{highlight}
  \begin{equation}
    \frac{1}{\theta}\mathbf q\cdot \mathbf g - \bm \sigma_D:\mathbf D\leq 0,
  \end{equation}
\end{highlight}
where $\mathbf D\equiv\operatorname{sym}(\mathbf L) = \operatorname{sym}(\dot{\mathbf F}\mathbf F^{-1})$ is the rate of deformation tensor.
The Navier-Stokes fluid is an example of this type of material and its constitutive equation for the stress is
\begin{equation}
  \bm \sigma = -p\mathbf I + 2\nu \mathbf D + \lambda (\operatorname{tr}\mathbf D).
\end{equation}

Another approach, concerning materials with fading memory is presented by Coleman \citep{colemanFoundationsLinearViscoelasticity1961, colemanThermodynamicsStrainImpulses1964, colemanThermodynamicsMaterialsMemory1964}.
It is based on the experimental observation that a material's response is more strongly influenced by recent events than by older events.
This assumption is made precise through an influence function, which characterizes the rate at which the memory fades.

In the approach of thermodynamics with internal variables, introduced by Coleman \citep{colemanThermodynamicsInternalState1967},
the thermodynamic state is assumed to be completely defined by the instantaneous values of a finite number of state variables
\begin{equation}
    \{\mat F, \theta, \vect g, \vect \alpha\}.
\end{equation}
at a given instant of the calorodynamic process, where
\begin{equation}
    \vect \alpha = \{\alpha_k\}
\end{equation}
is a set of internal variables, scalar or tensorial in nature, associated with dissipative mechanisms.
As such, the accuracy of the constitutive model depends strongly on the appropriate choice of internal variables, as these contain the relevant information about the thermodynamical history of the material.

Accordingly, the specific Helmholtz free energy is postulated to follow
\begin{highlight}
    \begin{equation}
        \psi = \psi(\mat F, \theta, \vect \alpha).
    \end{equation}
\end{highlight}
The constitutive equations for the Cauchy stress and entropy are found in the same way through the Clausius-Duhem inequatlity, and found to be equal to Equations~\eqref{eq:constitutive_equation_stress_thermoelasticity} and \eqref{eq:const_entropy}.

For each internal variable $\alpha_k$ of the set $\alpha$ of internal variables, the conjugate thermodynamical forces are defined to be
\begin{equation}
    A_k\equiv \rho_0 \frac{\partial \psi}{\partial \alpha_k},
\end{equation}
so that the Clausius-Duhem equation can be written in a reduced form as
\begin{highlight}
    \begin{equation}
        -\bm A*\dot{\vect \alpha} - \frac{J}{\theta}\vect q\cdot \vect g\geq 0,
    \end{equation}
\end{highlight}
where $\vect A$ is the set of conjugate thermodynamical forces and \(*\) denotes the appropriate product operation.

To completely define the constitutive model, one still needs to postulate the constitutive equations for the flux variables $\dot{\vect \alpha}$ and $\frac{1}{\theta}\vect q$.
These are given by
\begin{highlight}[innertopmargin=-5pt]
            \begin{gather}     \dot{\vect \alpha} = f(\mat F, \theta, \vect g, \vect \alpha),\\      \frac{1}{\theta}\vect q = g(\mat F, \theta, \vect g, \vect \alpha).
            \end{gather}
\end{highlight}

Employing the same rational as before the internal dissipation is found to be
\begin{equation}
  \pazocal D_\text{int} = -\mathbf A *\dot{\bm \alpha}.
\end{equation}
A sufficient condition for the internal dissipation to be non negative is the hypothesis of normal dissipativity \citep{desouzanetoComputationalMethodsPlasticity2008}.
Here, a non-zero dissipation is possible, implying a large perturbation away from thermodynamic equilibrium.
Thus, this approach finds use to describe for example rate-independent plasticity \citep{desouzanetoComputationalMethodsPlasticity2008}.
A connection between infinitesimal viscoelasticity with the general approach of thermodynamics with internal variables is made clear in Section~\ref{sec:infinitesimal_thermo_viscoelasticity}.

\newpage\null\thispagestyle{blank}\newpage
