\chapter{Numerical results} \label{sec:numerical_results}

\section{Wishlist}

Independent variables
\begin{itemize}
  \item strain rate, temperature
  \item stress, temperature
\end{itemize}

Loading type
\begin{itemize}
  \item uniaxial traction
  \item uniaxial compression
  \item pure shear
  \item biaxial traction
  \item hydrostatic compression
  \item torsion
\end{itemize}

time dependency
\begin{itemize}
  \item impulse
  \item constant
  \item ramp
  \item harmonic
  \item cyclic
\end{itemize}

The most interesting loading schemes are:
\begin{itemize}
  \item constant strain rate;
  \item stress relaxation
  \item creep
  \item jump constant strain rate at different strains
  \item loading unloading
  \item free shrinkage
\end{itemize}


\section{Constant strain rate experiments}

The experiments employed will be.

Compare yield stresses.

Verify thermal sofetening.

Try and choose polymers above and below the glass transition temperature. Need to model the intrinsic softening.

Check how the initial stiffness varies with the strain rate and the temperature.

\section{Unloading}

Remaining deformation Strobl and friends

\section{Free shrinkage}

Remaining deformation

\section{Heated shrinkage}

\colorbox{CyanBlue}{Try the self healing stuff for fibers.}

\section{Stress relaxation}

Test with stress relaxation, maybe master curve.

\section{Creep}

Test with creep, maybe master curve.
